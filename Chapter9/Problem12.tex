\documentclass{article}
\usepackage{amsmath}
\usepackage{amsfonts}
\usepackage{amssymb}

\newenvironment{proof}{\paragraph{Proof:}}{\hfill$\square$}
\newtheorem{theorem}{Theorem}
\newtheorem{lemma}[theorem]{Lemma}
\newtheorem{corollary}[theorem]{Corollary}

\author{Arthur Chen}
\title{Problem 12}
\date{\today}

\begin{document}

\section*{Problem 12}

Let $X_1 \dots X_n$ be iid from an exponential distribution with $f(x|\theta) = \theta e^{-\theta x}, x \geq 0$. Derive a likelihood test for $H_0: \theta = \theta_0$ versus $H_A: \theta \neq \theta_0$ and show that the rejection region is of the form $\bar{X} e^{-\theta_0 \bar{X}} \leq c$.

The likelihood ratio is

\[
\Lambda = \frac{\max_{\theta = \theta_0} lik(\theta)}
{\max_{\theta \in \mathbb{R^+}} lik(\theta)}
\]

where $lik(\theta|x) = \theta^n e^{-\theta \sum X_i}$. The maximum likelihood estimator of an exponential distribution is $\theta^* = \frac{1}{\bar{X}}$. Since small values of $\Lambda$ are evidence against the null, the test rejects the null when

\[
\Lambda = \frac{\theta_0^n e^{-\theta_0 n \bar{X}}}
{(\frac{1}{\bar{X}})^n e^n} < c
\]

$\theta_0^n$ is a constant and can be immediately absorbed into c. For the exponent, $\bar{X}^n e^{-\theta_0 n \bar{X} - n} = (\bar{X}e^{-\theta_0 \bar{X}e^{-1}})^n$, and since $\Lambda$ is positive, taking the nth root won't introduce more inequalities. Thus by taking the nth root of both sides, $e^{-1}$ can be absorbed into c, the rejection region has the form

\[
\bar{X} e^{-\theta_0 \bar{X}} < c
\]

as desired.

\end{document}